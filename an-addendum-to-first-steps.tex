\documentclass{amsart}
\usepackage{hyperref}
\usepackage{amsmath}
\usepackage{amssymb}
\usepackage{amsthm}
\usepackage{tikz-cd}
\usepackage{natbib}

% Theorems
\newtheorem{thm}{Theorem}
\newtheorem{cor}[thm]{Corollary}
\newtheorem{lem}[thm]{Lemma}
\newtheorem{remark}[thm]{Remark}
\newtheorem{example}[thm]{Example}
\newtheorem{defn}[thm]{Definition}

% Categories stuff
\newcommand{\Ccat}{\ensuremath{\mathbb{C}}}
\newcommand{\Dcat}{\ensuremath{\mathbb{D}}}
\newcommand{\op}[1]{\ensuremath{#1^{\mathsf{op}}}}
\newcommand{\SET}{\ensuremath{\mathbf{Set}}}
\newcommand{\presheaves}[1]{\ensuremath{\widehat{#1}}}
\newcommand{\transpose}[1]{\ensuremath{\widehat{#1}}}
\newcommand{\symmetrize}[1]{\ensuremath{\widetilde{#1}}}
\newcommand{\pair}[2]{\ensuremath{\left\langle #1, #2 \right\rangle}}
\newcommand{\fix}[1]{\ensuremath{\mathsf{fix}(#1)}}
\newcommand{\lam}[2]{\ensuremath{\lambda #1.\,#2}}
\makeatletter
\def\rightharpoonupfill@{\arrowfill@\relbar\relbar\rightharpoonup}
\newcommand{\arr}{\mathpalette{\overarrow@\rightharpoonupfill@}}
\makeatother


\title{An Addendum to Section 7 of \emph{First Steps in Synthetic
    Guarded Domain Theory}}
\author{Daniel Gratzer}
\date{\today}

\begin{document}
\maketitle

The following is a brief explanation of how to fill a small gap in the
proof of Theorem 7.5 in \citet{Birkedal:11} that I couldn't originally
see how to fill. The crux of the issue is that when symmetrizing a
functor which takes two arguments:
$F : \Dcat \times (\op{\Ccat} \times \Ccat) \to \Ccat$, it is only
reasonable to set $\symmetrize{F}$ to the following type.
\[
  \symmetrize{F} : \op{\Dcat} \times \Dcat \times \op{\Ccat} \times \Ccat \to \op{\Ccat} \times \Ccat
\]
For which the definition of the object half of the new functor is:
\[
  \symmetrize{F}(D_1, D_2, C_1, C_2) = (F(D_1, C_2, C_1), F(D_2, C_1, C_2))
\]
This alteration means that the lemmas provided in \citet{Birkedal:11}
must be generalized slightly to handle this.
\begin{lem}[Fixed Points are Initial]\label{lem:initial}
  If $F$ is a locally contractive functor $F : \Ccat \to \Ccat$ and
  $I$ is an object such that $\iota : F(I) \cong I$ then
  $\iota : F(I) \to I$ forms an initial algebra for $F$.
\end{lem}
\begin{proof}
  In order to show that this algebra is initial, suppose we have some
  $g : F(A) \to A$. We define the unique algebra homomorphism as the
  unique fixed point to the following contractive function.
  \[
    h \mapsto g \circ F(h) \circ \iota^{-1}
  \]
  This map is contractive by the assumption that the morphism
  component of $F$ is witnessed by contractive maps. This is obviously
  an algebra homomorphism. Any algebra homomorphism, $h'$, must
  satisfy the equation
  \[
    h' \circ \iota = g \circ F(h') \iff
    h' = g \circ F(h') \circ \iota^{-1}
  \]
  Therefore any algebra homomorphism is a fixed point of the above
  equation and by the uniqueness of fixed points $h' = h$.
\end{proof}
\begin{lem}\label{lem:internal-choice}
  The choice of initial maps from Lemma~\ref{lem:initial} may be given
  as a family of maps. Explicitly, if $\iota : F(I) \cong I$ then, for
  any $Z$, there is a $f_z : Z^{F(Z)} \to Z^I$ satisfying the
  following formula internally.
  \[
    \forall g : Z^{F(Z)}.\ f_z(g) \circ \iota = g \circ F(f_z(g))
  \]
\end{lem}
\begin{proof}
  The construction of these families of maps is trivial. The map for some
  $Z$ is given by
  \[
    f_z = \lam{g : Z^{F(Z)}}{\fix{\lam{h : Z^I}{g \circ F(h) \circ \iota^{-1}}}}
  \]
  The formula follows in the internal language by the fact that in the internal language we have
  that a map $\fix{-}$ such that $\fix{f} = f(\fix{f})$ when $f$ is contractive. This map may be
  constructed using unique choice, fixed points are unique so we can construct a functional from the
  internal property that they exist (a proof that unique choice holds in any topos may be found in
  Theorem II.5.9 of \citet{Lambek:88}). The proof that we have unique fixed points is given by
  Theorem 2.9~\citep{Birkedal:11}.
\end{proof}
We now turn to the part which actually differs. A number of the small
lemmas given in the paper must be given explicitly with regards to a
parameterization in order to be sufficiently general to compose in the
final proof.
\begin{lem}[Parameterized Initial Algebras]\label{lem:parameterized-initial-algebra}
  If $F : \Dcat \times \op{\Ccat} \times \Ccat \to \Ccat$, then for
  any $A, B \in \Dcat$ if there is an $X, Y \in \Ccat$ such that
  \[
    \iota_2 : F(A, X, Y) \cong Y \qquad \iota_1 : F(B, Y, X) \cong X
  \]
  Then $(X, Y)$ is an initial algebra of
  $\symmetrize{F}(A, B, -, -) : \op{\Ccat} \times \Ccat \to \op{\Ccat}
  \times \Ccat$. Furthermore, if $A \cong B$ then $X \cong Y$.
\end{lem}
\begin{proof}
  Under these assumptions we may construct an isomorphism
  $(\iota_1, \iota_2) : \symmetrize{F}(A, B, X, Y) \to (X, Y)$. By
  Lemma~\ref{lem:initial} then this tells us that $(\iota_1, \iota_2)$
  is an initial algebra. Moreover, if $A \cong B$ then we may easily
  construct an initial algebra structure on $(Y, X)$ such that in this
  case $X \cong Y$.
\end{proof}
\begin{lem}\label{lem:flipped-initial-algebras}
  For all
  $F : \Dcat \times \op{\Ccat} \times \Ccat \to \Ccat$,
  if for some $A, B \in \Dcat$ there is an $X, Y \in \Ccat$ such that
  $(X, Y)$ is an initial algebra for $\symmetrize{F}(A, B, -, -)$, then
  $(Y, X)$ is an initial algebra for $\symmetrize{F}(B, A, -, -)$ in
  $\Ccat \times \op{\Ccat}$.
\end{lem}
\begin{proof}
  First, if $(X, Y)$ is an initial algebra for
  $\symmetrize{F}(A, B, -, -)$ then we have $i_1 : X \to F(A, Y, X)$
  and $i_2 : F(B, X, Y) \to Y$ satisfying the appropriate universal
  property.

  An algebra for $\symmetrize{F}(B, A, -, -)$ in
  $\op{\Ccat} \times \Ccat$ is some $(W, Z)$ equipped with a map
  $g : \symmetrize{F}(B, A, W, Z) \to (W, Z)$. That is, a map
  $g_1 : F(A, Z, W) \to W$ and $g_2 : Z \to F(B, W, Z)$. This is
  precisely the data required for an algebra for $(Z, W)$ on
  $\symmetrize{F}(A, B, -, -)$ in $\op{\Ccat} \times \Ccat$ so there's
  a unique map $(h_1, h_2) : (X, Y) \to (Z, W)$. It is easy to observe
  that $(h_2, h_1) : (Y, X) \to (W, Z)$ in $\Ccat \times \op{\Ccat}$
  and that this constitutes an algebra homomorphism for
  $F(B, A, -, -)$. Uniqueness is given by the fact that all algebra
  homomorphisms arise in this way.
\end{proof}
This lemma goes through unchanged.
\begin{lem}\label{lem:mu-construction}
  Suppose that $F : \Dcat \times \Ccat \to \Ccat$ and for any
  $D \in \Dcat$ that $F(D, -)$ has an initial algebra,
  $f_D : F(D, I_D) \to I_D$, then there exists a functor
  $\mu F : \Dcat \to \Ccat$ such that for all $D \in \Dcat$, $\mu F(D)$
  is the initial algebra for $F(D, -)$ (given by an appropriate family
  of morphisms internally witnessing initiality as in
  Lemma~\ref{lem:internal-choice}).

  Moreover, if $F$ is contractive in $\Dcat$ then so is $\mu F$.
\end{lem}
\begin{proof}
  Define the object part of $(\mu F)(D) = I_D$. Then, for a morphism
  $g : D \to D'$ define $(\mu F)(g)$ to be the unique algebra
  homomorphism fitting into the diagram below.
  \[
    \begin{tikzcd}
      F(D, I_D) \ar[dd, swap, "f_D"] \ar[r, "{F(D, h)}"] & F(D, I_{D'}) \ar[d, "{F(g, I_{D'})}"]\\
      & F(D', I_{D'}) \ar[d, "f_{D'}"]\\
      I_D \ar[r, swap, "h"] & I_{D'}
    \end{tikzcd}
  \]
  Then this gives the desired morphism $(\mu F)(g) = h$. The
  functoriality of this choice is a direct result of the fact that $h$
  is unique with the property of being a homomorphism between these
  two algebras.

  The morphism part of this functor can be expressed as an indexed
  family of maps by composing the given family of maps for choosing
  initial algebras with the map $Y^X \to I_Y^{F(X, I_Y)}$ using the
  morphisms associated with $F$ as well as $f_Y$. This choice is
  contractive if $F$ is contractive since we compose with the strength
  of $F$.
\end{proof}

\begin{lem}[Unparameterized Mixed Variance Fixed Points]\label{lem:unparameterized}
  For any $F : \Dcat \times \op{\Ccat} \times \Ccat \to \Ccat$ and
  $A, B \in \Dcat$, if $\Ccat$ has solutions to contractive functors
  then there exists an $X$ and a $Y$ such that
  \[
    F(A, X, Y) \cong Y \qquad F(B, Y, X) \cong X
  \]
\end{lem}
\begin{proof}
  Define $G = F(A, -, -) : \op{\Ccat} \times \Ccat \to \Ccat$ and also
  define $H = \mu F(B, -, -) : \op{\Ccat} \to \Ccat$ using the
  construction given in Lemma~\ref{lem:mu-construction} sending each
  $C \in \op{\Ccat}$ to the fixed point of $F(B, C, -)$.

  Set $X$ as the fixed point of $C \mapsto G(H(C), C)$ (here viewing
  $H : \Ccat \to \op{\Ccat}$) and set $Y = H(X)$. Then we have that
  $X \cong G(H(X), X) = F(A, H(X), X) = F(A, Y, X)$ and
  $Y = H(X) \cong F(B, X, Y)$.
\end{proof}

\begin{thm}[Functorially Parameterized Fixed Points]
  Suppose that
  $F : (\op{\Ccat} \times \Ccat)^n \times (\op{\Ccat} \times \Ccat) \to \Ccat$
  and is contractive in the final pair of $\op{\Ccat} \times
  \Ccat$. Then there exists a functor $\fix{F}$ such that
  \[
    \fix{F} : (\op{\Ccat} \times \Ccat)^n \to \Ccat
  \]
  and such that
  \[
    F \circ \pair{1}{\symmetrize{\fix{F}}} \cong \fix{F}
  \]
  Moreover, if $F$ is contractive in all pairs then so is $\fix{F}$.
\end{thm}
\begin{proof}
  Set $\Dcat = (\op{\Ccat} \times \Ccat)^n$, then
  $F : \Dcat \times \op{\Ccat} \times \Ccat \to \Ccat$ and therefore
  Lemma~\ref{lem:unparameterized} applies to give us the preconditions
  to Lemma~\ref{lem:parameterized-initial-algebra} so we have a choice
  of initial algebras for $\symmetrize{F}(A, B, -, -)$,
  $(X_{A, B}, Y_{A, B})$ for any $A, B \in \Dcat$ such that if
  $A \cong B$ then $X_{A, B} \cong Y_{A, B}$. Recall that
  $\symmetrize{F}$ has the type:
  \[
    \symmetrize{F} :
    (\op{\Dcat} \times \Dcat) \times (\op{\Ccat} \times \Ccat) \to (\op{\Ccat} \times \Ccat)
  \]
  This is in the shape for Lemma~\ref{lem:mu-construction} where we
  view the $\Dcat$ of this lemma as our $\op{\Dcat} \times \Dcat$ and
  $\Ccat$ as our $\op{\Ccat} \times \Ccat$. This gives us a functor
  \[
    \mu \symmetrize{F} :
    \op{\Dcat} \times \Dcat \to \op{\Ccat} \times \Ccat
  \]
  This functor is equipped with the property that
  $\mu \symmetrize{F}(A, B) = (X_{A, B}, Y_{A, B})$. Observe that since
  $\Dcat = (\op{\Ccat} \times \Ccat)^n$ there is a functor
  aspirationally named $\Delta : \Dcat \to \op{\Dcat} \times \Dcat$
  defined by
  \[
    \Delta(\arr{C^o}, \arr{C}) = ((\arr{C}, \arr{C^o}), (\arr{C^o}, \arr{C}))
  \]
  Finally, compose with the $\Delta$ and the projection:
  \[
    \fix{F} = \pi_2 \circ \mu \symmetrize{F} \circ \Delta
    : \Dcat \to \Ccat
  \]
  To get the final property desired, suppose we have some
  $D = (\arr{C^o}, \arr{C}) \in (\op{\Ccat} \times \Ccat)^n$. As a
  small piece of notation, write $\op{D}$ for $(\arr{C}, \arr{C^o})$.
  Then notice that (if we denote the initial algebra constructed above
  for $\symmetrize{F}(\op{D}, D, -, -)$ as $(X, Y)$) the following
  holds.
  \begin{align*}
    \fix{F}((\arr{C^o}, \arr{C}))
    &= (\pi_2 \circ \mu \symmetrize{F} \circ \Delta)(D)\\
    &= (\pi_2 \circ \mu \symmetrize{F})(\op{D}, D)\\
    &= \pi_2(X, Y) && \text{$\mu \symmetrize{F}$ is the initial
                      algebra at every point}\\
    &= Y
  \end{align*}
  Now on the other hand, if we consider the left-hand side of the
  equation we have
  \begin{align*}
    (F \circ \pair{1}{\symmetrize{\fix{F}}})(\arr{C^o}, \arr{C})
    &= F(D, \fix{F}(\op{D}), \fix{F}(D))\\
    &= F(D, \fix{F}(\op{D}), Y)\\
    &\cong F((\arr{C^o}, \arr{C}), X, Y)  && \text{This step is justified below.}\\
    &\cong Y
  \end{align*}
  With the last inference following by the results of
  Lemma~\ref{lem:unparameterized} which was used above to construct
  $X$ and $Y$. We have also made use of here that if $(X, Y)$ is an
  initial algebra for $\symmetrize{F}(\op{D}, D, -, -)$ viewed as
  $\op{\Ccat} \times \Ccat \to \op{\Ccat} \times \Ccat$ then $(Y, X)$
  is an initial algebra for $\symmetrize{F}(D, \op{D}, -, -)$ viewed
  as $\Ccat \times \op{\Ccat} \to \Ccat \times \op{\Ccat}$. This
  follows precisely from
  Lemma~\ref{lem:flipped-initial-algebras}. Explicitly worked out:
  \begin{align*}
    \fix{F}(\op{D})
    &= (\pi_2 \circ \mu \symmetrize{F} \circ \Delta)(\op{D})\\
    &= (\pi_2 \circ \mu \symmetrize{F})(D, \op{D})\\
    &\cong \pi_2(Y, X) && \text{$\mu \symmetrize{F}(D, \op{D})$ is the initial
                      algebra in $\Ccat \times \op{\Ccat}$}\\
    &= X
  \end{align*}
  Note further that the second to last line is merely an isomorphism:
  this is because we are only guaranteed to choose some initial
  algebra, not $(Y, X)$ in particular. The uniqueness of initial
  algebras up to isomorphism is sufficient to determine the fact that
  we wanted however. One can view this as baking in the final line of
  reasoning for Lemma~\ref{lem:parameterized-initial-algebra} and it
  explains why we never made explicit use of this fact.

  This demonstrates the desired isomorphism.

  The fact that $\fix{F}$ is contractive in all pairs is a result of
  the fact that $\mu \symmetrize{F}$ is contractive in its first
  argument if $F$ is.
\end{proof}

\bibliographystyle{plainnat}
\bibliography{citations}
\end{document}
